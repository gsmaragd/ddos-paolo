\documentclass[sigconf,letterpaper]{acmart}
%\documentclass[sigconf,anonymous]{acmart}
%\documentclass[sigconf, anonymous, review]{acmart}
%\documentclass[10pt,sigconf,anonymous]{acmart}

%%
%% \BibTeX command to typeset BibTeX logo in the docs
\AtBeginDocument{%
  \providecommand\BibTeX{{%
    \normalfont B\kern-0.5em{\scshape i\kern-0.25em b}\kern-0.8em\TeX}}}

%% Rights management information.  This information is sent to you
%% when you complete the rights form.  These commands have SAMPLE
%% values in them; it is your responsibility as an author to replace
%% the commands and values with those provided to you when you
%% complete the rights form.
% Copyright
%\renewcommand\footnotetextcopyrightpermission[1]{} % removes footnote with conference info
%\setcopyright{none}
%\acmConference{The Web Conference 2020}{April 20-24, 2019}{Taipei}
%\acmConference{}{}{}

%\setcopyright{acmcopyright}
%\copyrightyear{2018}
%\acmYear{2018}
%\acmDOI{10.1145/1122445.1122456}

%% These commands are for a PROCEEDINGS abstract or paper.
%\acmConference[Woodstock '18]{Woodstock '18: ACM Symposium on Neural
%  Gaze Detection}{June 03--05, 2018}{Woodstock, NY}
%\acmBooktitle{Woodstock '18: ACM Symposium on Neural Gaze Detection,
%  June 03--05, 2018, Woodstock, NY}
%\acmPrice{15.00}
%\acmISBN{978-1-4503-9999-9/18/06}


%%% printfolios = false -> removes page numbers %%%
\settopmatter{printacmref=false, printccs=false, printfolios=false}


\usepackage{balance}


%\fancyhf{} % Remove fancy page headers 
%\fancyhead[C]{Anonymous submission \#9999 to ACM CCS 2019} % TODO: replace 9999 with your paper number
%\fancyfoot[C]{\thepage}
%\fancyfoot{}

%\usepackage{lastpage}

%\usepackage{cleveref}

\renewcommand{\topfraction}{1}
\renewcommand{\dbltopfraction}{1}
\renewcommand{\floatpagefraction}{1}
\renewcommand{\dblfloatpagefraction}{1}
\renewcommand{\textfraction}{0}
\usepackage{stfloats}

\newcommand{\ie}{i.e., \@}
\newcommand{\eg}{e.g., \@}
\newcommand{\Ie}{I.e., \@}
\newcommand{\Eg}{E.g., \@}
\newcommand{\cf}{cf. \@}
\newcommand{\Cf}{Cf. \@}
\newcommand{\etc}{etc. \@}
\newcommand{\perc}{\,\%}
\newcommand{\hn}[1]{\textsl{\textbf{#1}\mbox{}}}
\newcommand{\etal}{et al.}

%
\newcommand{\eat}[1]{}
\newcommand{\todo}[1]{\textcolor{red}{TODO: \emph{#1}}}
\newcommand{\nlnote}[1]{\textcolor{fashionfuchsia}{\small \bf [NL: #1]}} %Laoutaris
\newcommand{\gsnote}[1]{\textcolor{blue}{\small \bf [GS: #1]}}
\newcommand{\colorcell}{\cellcolor{blue!20}}
\newcommand{\srdjan}[1]{\noindent{\color{blue}{\bf \fbox{srj}} {\it#1}} }
\newcommand{\nikos}[1]{\noindent{\color{red}{\bf \fbox{nik}} {\it#1}} }
\newcommand{\cinote}[1]{\noindent{\color{orange}{\bf \fbox{cos}} {\it#1}} }
\newcommand{\updateme}[1]{\noindent{\color{red}{\bf \fbox{UPDATE ME!}} {\it#1}} }
\newcommand{\workinprogres}{\noindent{\Large{\color{magenta}{WORK IN PROGRESS. DO NOT EDIT THE SECTION FROM THIS POINT ON.}}}}
%
\newcommand{\new}[1]{\noindent{\color{orange}{\bf \fbox{NEW}} {\it#1}}}
\newcommand{\curlie}[1]{Curlie}
\newcommand{\curliedotorg}{\url{http://curlie.org}}
\newcommand{\web}{Web}
\newcommand{\webpage}{web page}
\newcommand{\webpages}{web pages}
\newcommand{\website}{web site}
\newcommand{\websites}{web sites}
\newcommand{\overfittedclassifier}{ba\-se\-li\-ne cla\-s\-si\-fier}  %{\textcolor{red}{groundtruth classifier}}
\newcommand{\balancedclassifier}{ba\-la\-nced cla\-s\-si\-fier} %{\textcolor{red}{groundtruth-expanded classifier}}
\newcommand{\curlieurls}{1,525,865}
\newcommand{\curliecategories}{344,227}
\newcommand{\curliedomains}{1,137,997}
\newcommand{\curliedomainsmultipleurls}{43,839} % or 44,390???
\newcommand{\customcheckmark}{\ding{52}}

\newcommand{\afblock}[1]{\noindent{\textbf{#1}}} 


%%%
\newcommand{\fra}{\tt CE1}
\newcommand{\dus}{\tt CE2}
\newcommand{\ham}{\tt CE3}
\newcommand{\muc}{\tt CE4}
\newcommand{\nyc}{\tt NA1}
\newcommand{\dfw}{\tt NA2}
\newcommand{\mad}{\tt SE1}
\newcommand{\mrs}{\tt SE2}
\newcommand{\lis}{\tt SE3}
\newcommand{\ist}{\tt SE4}
\newcommand{\pmo}{\tt SE5}
%%%

\newif\iflatexdiff\latexdifftrue
\latexdifffalse

\newif\ifrevisionlog\revisionlogtrue 
\revisionlogfalse

\iflatexdiff
	\usepackage{pdfpages}
\fi

\ifrevisionlog
	\include{revisionnotes-preamble}
\fi


\usepackage[normalem]{ulem}
% Srdjan: package conflict issue with symbols on Linux; had to add those two lines.
\usepackage{savesym}
\savesymbol{Bbbk}
\usepackage{tabularx}
\newcolumntype{L}[1]{>{\raggedright\arraybackslash}p{#1}}
\newcolumntype{C}[1]{>{\centering\arraybackslash}p{#1}}
\newcolumntype{R}[1]{>{\raggedleft\arraybackslash}p{#1}}
\usepackage{graphicx}
%\usepackage{subfig}
\usepackage{float}
%\usepackage{subfig}
\usepackage{enumitem}
\setlist{nolistsep}
\usepackage{graphicx}
\usepackage{epstopdf}
\usepackage{grffile}
\usepackage{amssymb}
\usepackage{amsmath}
\usepackage{multicol,multirow}
\usepackage{rotating,url}
\usepackage{enumitem}
\usepackage{color}
\usepackage{xspace}
\usepackage{ifpdf}
\usepackage{mdwlist}
\usepackage{colortbl}
\usepackage{caption}
\usepackage{amssymb}
\usepackage{booktabs}
\usepackage{multirow}
\usepackage{enumitem}
\usepackage{xfrac}
\usepackage{afterpage}% http://ctan.org/pkg/afterpage
\usepackage{pdflscape}
\usepackage{longtable}
\usepackage{hhline}
\usepackage{lscape}
\usepackage{pifont}
% \usepackage{times}
% \usepackage{authblk}
% \usepackage{cite}
\usepackage{subfigure}
%\usepackage{subcaption}
% \usepackage{tikz}
% \usetikzlibrary{trees}
\usepackage{verbatim}
% \usepackage[keeplastbox]{flushend}
% \usepackage{algorithm}
% \usepackage[noend]{algorithmic}
% \usepackage{subfigure}
% \usepackage[caption=false]{subfig}
% \usepackage{subcaption}
% \usepackage[ruled,vlined,linesnumbered]{algorithm2e}
% Added by Costas
% \usepackage{subcaption}
% \usepackage{xcolor}
\usepackage{arydshln} % dashed lines in tables


%% To avoid broken URLs error. Remember to remove before submission 
\hypersetup{draft}

%%
%% Submission ID.
%% Use this when submitting an article to a sponsored event. You'll
%% receive a unique submission ID from the organizers
%% of the event, and this ID should be used as the parameter to this command.
%%\acmSubmissionID{123-A56-BU3}

%%
%% The majority of ACM publications use numbered citations and
%% references.  The command \citestyle{authoryear} switches to the
%% "author year" style.
%%
%% If you are preparing content for an event
%% sponsored by ACM SIGGRAPH, you must use the "author year" style of
%% citations and references.
%% Uncommenting
%% the next command will enable that style.
%%\citestyle{acmauthoryear}

\fancyhead{}

%%
%% end of the preamble, start of the body of the document source.


%%%%% Permission Release %%%%%
%\copyrightyear{2021}
%\acmYear{2021}
%\setcopyright{rightsretained}
%\acmConference[CCS '21] {Proceedings of the 2021 ACM SIGSAC Conference on Computer and Communications %Security}{November 15--19, 2021}{Virtual Event, Republic of Korea.}
%\acmBooktitle{Proceedings of the 2021 ACM SIGSAC Conference on Computer and Communications Security (CCS %'21), November 15--19, 2021, Virtual Event, Republic of Korea}
%\acmPrice{}
%\acmISBN{978-1-4503-8454-4/21/11}
%\acmDOI{10.1145/3460120.3485385}
% Authors, replace the red X's with your assigned DOI string during the rightsreview eform process.
%
\settopmatter{printacmref=false}
\begin{document}
%\fancyhead{}
%%%%%%%%%%%%%%%%%




%%
%% The "title" command has an optional parameter,
%% allowing the author to define a "short title" to be used in page headers.
\title{Anatomy of a National Scrubbing Center}
\subtitle{Paper \#XXX}
%, \pageref{page:end_of_main_body} pages body, \pageref{page:last} pages total}

%%
%% The "author" command and its associated commands are used to define
%% the authors and their affiliations.
%% Of note is the shared affiliation of the first two authors, and the
%% "authornote" and "authornotemark" commands
%% used to denote shared contribution to the research.

% \author{First Second}
% \authornote{Both authors contributed equally to this research.}
% \email{trovato@corporation.com}
% \orcid{1234-5678-9012}
% \author{G.K.M. Tobin}
% \authornotemark[1]
% \email{webmaster@marysville-ohio.com}
% \affiliation{%
%   \institution{Institute for Clarity in Documentation}
%   \streetaddress{P.O. Box 1212}
%   \city{Dublin}
%   \state{Ohio}
%   \postcode{43017-6221}
% }

%\author{Paper \#XXX}

%\author{First Second}
%\affiliation{%
%  \institution{The Th{\o}rv{\"a}ld Group}
%}
%\email{second@affiliation.org}
%\author{First1 Second1}
%\affiliation{%
%  \institution{Inria Paris-Rocquencourt}
%}
%\email{second2@affiliation.org}

%%
%% By default, the full list of authors will be used in the page
%% headers. Often, this list is too long, and will overlap
%% other information printed in the page headers. This command allows
%% the author to define a more concise list
%% of authors' names for this purpose.

%\author{Name}
 %\affiliation{
% 	\institution{Institute}
 %}



%\renewcommand{\shortauthors}{} %Trovato and Tobin, et al.}

%%
%% The abstract is a short summary of the work to be presented in the
%% article.

% Disable / remove copyright boxes -- Do not move it upwards
\setcopyright{none}
\settopmatter{printacmref=false, printccs=false, printfolios=false}
\renewcommand\footnotetextcopyrightpermission[1]{}
\pagestyle{plain}
\acmConference{}{}{}
\renewcommand{\shortauthors}{}



\begin{abstract}

Abstract

\end{abstract}


%\begin{CCSXML}
%<ccs2012>
%<concept>
%<concept_id>10002978.10003014</concept_id>
%<concept_desc>Security and privacy~Network security</concept_desc>
%<concept_significance>500</concept_significance>
%</concept>
%</ccs2012>
%\end{CCSXML}

%\ccsdesc[500]{Security and privacy~Network security}
%\keywords{Cyberattacks, DDoS, amplification attacks, IXP}

%% The code below is generated by the tool at http://dl.acm.org/ccs.cfm.
%% Please copy and paste the code instead of the example below.
%\begin{CCSXML}
%<ccs2012>
% <concept>
%  <concept_id>10010520.10010553.10010562</concept_id>
%  <concept_desc>Computer systems organization~Embedded systems</concept_desc>
%  <concept_significance>500</concept_significance>
% </concept>
% <concept>
%  <concept_id>10010520.10010575.10010755</concept_id>
%  <concept_desc>Computer systems organization~Redundancy</concept_desc>
%  <concept_significance>300</concept_significance>
% </concept>
% <concept>
%  <concept_id>10010520.10010553.10010554</concept_id>
%  <concept_desc>Computer systems organization~Robotics</concept_desc>
%  <concept_significance>100</concept_significance>
% </concept>
% <concept>
%  <concept_id>10003033.10003083.10003095</concept_id>
%  <concept_desc>Networks~Network reliability</concept_desc>
%  <concept_significance>100</concept_significance>
% </concept>
%</ccs2012>
%\end{CCSXML}
%
%\ccsdesc[500]{Computer systems organization~Embedded systems}
%\ccsdesc[300]{Computer systems organization~Redundancy}
%\ccsdesc{Computer systems organization~Robotics}
%\ccsdesc[100]{Networks~Network reliability}

%%
%% Keywords. The author(s) should pick words that accurately describe
%% the work being presented. Separate the keywords with commas.
%\keywords{datasets, neural networks, gaze detection, text tagging}

%%
%% This command processes the author and affiliation and title
%% information and builds the first part of the formatted document.
\maketitle

\section{Introduction}\label{sec:intro}

scope and motivation

previous work.
In 2020, Moura et al.~\cite{Moura-scrubbing}

research questions

Our results can be summarized as follows:

\begin{itemize}

\item Analysis of a national-wide scrubbing center that serves a diverse set of networks, including commercial, government, and educational networks. %First analysis of a scrubbing center, I don't think this was ever done so it would be the first time that someone looks at scrubbing data in this way.

\item Characterization of more than XXX attacks for the period between September 2022 and February 2023. 
%Overall analysis and characteristics of attacks in 2022/2023

\item Our analysis shows that the duration of the attacks differs from this reported in the literature. 
We report than only half of the attacks ;ast for less than 10 minutes, while the longest are in the orders of hours or days.
 %Duration of attacks (Fig 6.8): only 50% are <10m the rest are much longer, this is different from what most papers say that 80% of attacks are ~10m long.

\item Our analysis also shows that the attackers are utilizing various attack vectors -- in some cases up to 15 attack vectors are utilized. 
%Distribution of attack vectors: the different attack vectors and also the
%number of vectors (up to 15) in the same attack is interesting, we might need
%to analyse both periods the same way because there would be clear differences
%otherwise.

\item We also report on the impact of taking down a booter in December 2022 as observed in the scrubbing center. 
%Study of the booter takedown of Dec 2022 with weekly frequency of attacks.


%<I think it is better to mention this in the results section when we discuss it> Spoofing analysis? We don't have data from recent attacks. The analysis
%for .nl is interesting (Fig 4.2) and the fact that the method could be used to
%detect spoofing using the TTL distribution.
%
%%\item Do we have results for the packet size etc.?

\end{itemize}


\section{Scrubbing Center Datasets}\label{sec:datasets}


\subsection{Scrubbing Center Profile}


\subsection{Network and Attack Datasets}

We have two datasets from NBIP:

Feb 2022 - Jun 2022: 1400 attacks in sflow format, the sampling rate is between 1:8129 and 1:64000. We used DDoS Dissector to analyze these and get the corresponding fingerprints.

%> Do you think we can ask Pim for the data between Jun 2022-Sep 2022, e.g.,
%> NBIP reports, so we do not have a gap. Should we also ask Pim for data from
%> February 2023 and later, e.g., until end of 2023?
%I think we can ask, getting the data should be straight forward for them since
%they have a database with all the entries. I think the most interesting period
%would be Jun 2022 - Jun 2023, covering before and after the takedowns.

Sep 2022 - Feb 2023: NBIP reports, they don't collect sflows unless needed so they sent us their reports which are similar to the DDoS Dissector fingerprints but with less information.

% > Do you think that we should cover both or only focus on the NBIP reports
% (Sep 2022-later)?
%The NBIP reports are more reliable since they create them with the full
%traffic/real-time, I would probably use the sflows to confirm what the report
%is saying due to the additional information we might find there.

%> Are there other takedowns or other events we can investigate?
%Yes, there seemed to be more arrests in May 2023 linked to the same takedown of
%December 2022:
%https://arstechnica.com/information-technology/2023/05/feds-seize-13-more-ddos-for-hire-platforms-in-ongoing-international-crackdown/
%It also seems like there is a team of people behind this working with the FBI:
%https://www.wired.com/story/big-pipes-ddos-for-hire-fbi/



\section{Discussion}\label{sec:discussion}


\section{Analysis}\label{sec:analysis}



\section{Conclusion}\label{sec:conclusion}



%\section{Scrubbing Center Datasets}\label{sec:datasets}


\subsection{Scrubbing Center Profile}


\subsection{Network and Attack Datasets}

We have two datasets from NBIP:

Feb 2022 - Jun 2022: 1400 attacks in sflow format, the sampling rate is between 1:8129 and 1:64000. We used DDoS Dissector to analyze these and get the corresponding fingerprints.

%> Do you think we can ask Pim for the data between Jun 2022-Sep 2022, e.g.,
%> NBIP reports, so we do not have a gap. Should we also ask Pim for data from
%> February 2023 and later, e.g., until end of 2023?
%I think we can ask, getting the data should be straight forward for them since
%they have a database with all the entries. I think the most interesting period
%would be Jun 2022 - Jun 2023, covering before and after the takedowns.

Sep 2022 - Feb 2023: NBIP reports, they don't collect sflows unless needed so they sent us their reports which are similar to the DDoS Dissector fingerprints but with less information.

% > Do you think that we should cover both or only focus on the NBIP reports
% (Sep 2022-later)?
%The NBIP reports are more reliable since they create them with the full
%traffic/real-time, I would probably use the sflows to confirm what the report
%is saying due to the additional information we might find there.

%> Are there other takedowns or other events we can investigate?
%Yes, there seemed to be more arrests in May 2023 linked to the same takedown of
%December 2022:
%https://arstechnica.com/information-technology/2023/05/feds-seize-13-more-ddos-for-hire-platforms-in-ongoing-international-crackdown/
%It also seems like there is a team of people behind this working with the FBI:
%https://www.wired.com/story/big-pipes-ddos-for-hire-fbi/



%\input{sections/ethics}
%\input{sections/anatomy}
%\section{Data Sets and Detection Approach}
%\input{sections/detection}
%\input{sections/characterization-challenges}
%\input{sections/characterization-opportunities}
%\input{sections/characterization-ixp}
%\section{How Distributed are DDoS Attacks?}
%\section{Distributed Sensing of DDoS Attacks}
%\input{sections/setting}
%\input{sections/characterization-attacks}
%\input{sections/early-detection}
%\input{sections/mitigation}
%\input{sections/related}


%\input{sections/acknowledgements}

%%
%% The next two lines define the bibliography style to be used, and
%% the bibliography file.
%\bibliographystyle{acm}
\balance
\bibliographystyle{ACM-Reference-Format}
\bibliography{paper}



%%%%% APPENDIX %%%%%%%
%\clearpage %Force appendix in a new page after references
%\section*{Appendix}
%\appendix
%\setcounter{secnumdepth}{0}
%\section{Appendix} \label{sec:appendix}

%\input{sections/feature-set}
%\input{sections/boosting-factor}




%\label{page:last}


 \ifrevisionlog
 	\newpage
 	\include{revisionnotes}
 \fi
 
% \iflatexdiff
% 	\newpage
% 	\includepdf[pages=-]{diff.pdf}%
% \fi


\end{document}
\endinput

%%% Local Variables:
%%% TeX-master: "paper"
%%% End:

