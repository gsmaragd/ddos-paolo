\section{Introduction}\label{sec:intro}

scope and motivation

previous work.
In 2020, Moura et al.~\cite{Moura-scrubbing}

research questions

Our results can be summarized as follows:

\begin{itemize}

\item Analysis of a national-wide scrubbing center that serves a diverse set of networks, including commercial, government, and educational networks. %First analysis of a scrubbing center, I don't think this was ever done so it would be the first time that someone looks at scrubbing data in this way.

\item Characterization of more than XXX attacks for the period between September 2022 and February 2023. 
%Overall analysis and characteristics of attacks in 2022/2023

\item Our analysis shows that the duration of the attacks differs from this reported in the literature. 
We report than only half of the attacks ;ast for less than 10 minutes, while the longest are in the orders of hours or days.
 %Duration of attacks (Fig 6.8): only 50% are <10m the rest are much longer, this is different from what most papers say that 80% of attacks are ~10m long.

\item Our analysis also shows that the attackers are utilizing various attack vectors -- in some cases up to 15 attack vectors are utilized. 
%Distribution of attack vectors: the different attack vectors and also the
%number of vectors (up to 15) in the same attack is interesting, we might need
%to analyse both periods the same way because there would be clear differences
%otherwise.

\item We also report on the impact of taking down a booter in December 2022 as observed in the scrubbing center. 
%Study of the booter takedown of Dec 2022 with weekly frequency of attacks.


%<I think it is better to mention this in the results section when we discuss it> Spoofing analysis? We don't have data from recent attacks. The analysis
%for .nl is interesting (Fig 4.2) and the fact that the method could be used to
%detect spoofing using the TTL distribution.
%
%%\item Do we have results for the packet size etc.?

\end{itemize}

